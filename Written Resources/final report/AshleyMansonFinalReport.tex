%----------------------------------------------------------------------------
% Start
%----------------------------------------------------------------------------

\documentclass{article}

\usepackage{graphicx} % Required for the inclusion of images
\usepackage{tocloft}
\renewcommand{\cftsecleader}{\cftdotfill{\cftdotsep}}
\renewcommand{\labelenumi}{\alph{enumi}.}

\setlength\parindent{0pt} % Removes all indentation from paragraphs


%----------------------------------------------------------------------------
% Document Information
%----------------------------------------------------------------------------

\begin{titlepage}

\title{Giguesaur: Game Logic}
\author{Ashley Manson}
\date{\today}

\begin{document}
\maketitle

\begin{center}
Co-Workers: Joshua La Pine \& Shahne Rodgers\\
Supervisors: Geoff Wyvill \& David Eyers\\

\vspace*{1\baselineskip} % Skip a line

Dept. of Computer Science\\
University of Otago
\end{center}

\end{titlepage}

%----------------------------------------------------------------------------
% Table of Contents
%----------------------------------------------------------------------------

\tableofcontents
\newpage

%----------------------------------------------------------------------------
% Introduction
%----------------------------------------------------------------------------

\section{Introduction}

Our vision for our completed Giguesaur application was allowing a classroom of children, each with their their own iPad, to run around and solve a jigsaw puzzle together. Imagine a classroom full of kids where they are all trying to work on a single conventional jigsaw puzzle; such a scheme is in no way pratical. The main goal of our project, besides all the design and technical subgoals, is simply to make a that is fun for children to play and work together.

% Overview of the project
\subsection{Overview}
The Giguesaur application development was divided into three different components. Joshua La Pine was in charge of developing the computer vsion part of the project, which allows for the puzzle pieces to be rendered over top the 'game board' in the real world. Shahne Rodgers took charge of the networking component of the project, which was crucial in allowing more than one player to interact with the jigsaw puzzle. Finally my part of the project was to develop the game logic and render the game to the iPad's screen.

% Background including what a puzzle and other games
\subsection{Background}

% Brief introduction on my work
\subsection{Game Logic}
As I stated previously, I was in charge of developing the game logic for the Giguesaur game.

%---------------------------------------------------------------------------
% Work Done
%---------------------------------------------------------------------------

\section{Work Done}

% Work done on Mac
\subsection{Start of Development}

% This is the build that had perspective stuff
\subsection{Prototype}

% Porting to iPad including changes and problems incountered
\subsection{Port to iPad}

% Integrating with others including problems incountered
\subsection{Integration}

%---------------------------------------------------------------------------
% Conclusion
%---------------------------------------------------------------------------

\section{Conclusion}

% What the thing looks like
\subsection{Final Build}

% What hasn't been done
\subsection{Future Work}

%\begin{figure}[h]
%\begin{center}
%\includegraphics[width=0.65\textwidth]{placeholder} % Include the image placeholder.png
%\caption{Figure caption.}
%\end{center}
%\end{figure}


%---------------------------------------------------------------------------
% References
%---------------------------------------------------------------------------

\pagebreak
\section{References}
\bibliographystyle{ieeetr}
\bibliography{sample}

%---------------------------------------------------------------------------
% End
%---------------------------------------------------------------------------

\end{document}
